\documentclass[10pt, letterpaper]{deps1}
%
% Preamble
%
\usepackage{graphicx}

%
% Document Environment
%
\begin{document}

% last update
\lastupdated

%
% Name
%
\begin{center}
    \textbf{\fontsize{40}{40}\selectfont Viktor Khristenko}
\end{center}

%
% Personal Entries
%
\vspace{-5pt}
\begin{spacing}{1}
\PersonalEntry{\textbullet{} Address}{CERN, 513/2-015, CH-1211, Geneva 23, Switzerland}
\PersonalEntry{\textbullet{} Phones}{ +33768966437 \textbullet{} +7 910 748 15 14}
\PersonalEntry{\textbullet{} Email}{\href{vdkhristenko1991@gmail.com}{vdkhristenko1991@gmail.com}}
\PersonalEntry{\textbullet{} Linkedin}{\urlstyle{same}\url{https://www.linkedin.com/in/viktor-khristenko}}
\PersonalEntry{\textbullet{} Github}{\urlstyle{same}{https://github.com/vkhristenko}}
\PersonalEntry{\textbullet{} \textbf{Languages}}{Russian(native) \textbullet{} English(native fluency) \textbullet{} French(B1)}
\end{spacing}

%
% Education
%
\vspace{4pt}
%\pagebreak

%\EducationEntry{\textbf{Online Education Certificates}}{}{Coursera}{
%    \vspace{-10pt}
%    \begin{tightitemize}
%        \item \CreateKeyword{Functional Programming} Principles in \CreateKeyword{Scala} (EPFL). \hfill\href{https://www.coursera.org/account/accomplishments/records/QQ4745GU8JBF}{Certificate}
%        \item \CreateKeyword{Functional Program} Design in \CreateKeyword{Scala} (EPFL). \hfill\href{https://www.coursera.org/account/accomplishments/records/PDGK52SUK4CN}{Certificate}
%        \item \CreateKeyword{Machine Learning} Foundations: A Case Study Approach (University of Washington) \hfill\href{https://www.coursera.org/account/accomplishments/records/WMVEKAAVEMDK}{Certificate}
%        \item \CreateKeyword{Machine Learning}: Regression (University of Washington) \hfill\href{https://www.coursera.org/account/accomplishments/records/F7PRDGN3PASK}{Certificate}
%        \item \CreateKeyword{Machine Learning}: Classification (University of Washington) \hfill\href{https://www.coursera.org/account/accomplishments/records/ZLKVPJM2QG4L}{Certificate}
%    \end{tightitemize}
%}

%
% Professional Experience
%
%\vspace{4pt}
\Section{Professional Activities}
\ExperienceEntry{\textbf{Software Engineer}}{2017 Sep - Current}{\CreateKeyword{CERN} - European Organization for Nuclear Research, Geneva, Switzerland}{
    \vspace{-10pt}
    \begin{tightitemize}
        \item \CreateKeyword{DEEP-EST} Project Member - EU initiative to build a modular supercomputing architecture for \CreateKeyword{exascale}
        \item \CreateKeyword{Large Scale HEP Data Processing} with Apache Spark. \url{https://www.youtube.com/watch?v=JTMdkSmw0Kc} Implemented Apache Spark's Data Source for ROOT file format. The data source was successfully used processing TBs+ datasets in various environments (e.g. cloud and HPC resources)
        \item \CreateKeyword{Heterogeneous Computing for CMS Experiment}. Modern HPC facilities draw enormous computing power from various accelerators. Within the DEEP-EST project, CMS Hadron and Electromagnetic Calorimeters workflows (CPU only) were ported to CUDA and optimized to target Nvidia V100 GPUs. Minimal viable reproducer was also ported to OpenCL (with FPGA specific extensions) and evaluated using Intel Arria 10 FPGA.
        \item \CreateKeyword{Employing various HPC resources} (e.g. JSC, Flatiron Institute) for HEP data processing
    \end{tightitemize}
} \\
\ExperienceEntry{\textbf{Group Lead - CMS Hadron Calorimeter Data Quality Monitoring Group}}{2014-2017}{\CreateKeyword{CERN} - European Organization for Nuclear Research, Geneva, Switzerland}{
    \vspace{-10pt}
    \begin{tightitemize}
        \item Designed and Implemented Critical \CreateKeyword{Data-driven Quality Control Applications}
        \item Tuned the runtime performance of applications to target Online CMS/LHC data taking conditions (40MHz collision rate)
    \end{tightitemize}
} \\
\ExperienceEntry{\textbf{Deputy Coordinator - CMS Hadron Calorimeter Operations Group}}{2015-2016}{\CreateKeyword{CMS} - Compact Muon Solenoid Experiment @CERN, Geneva, Switzerland}{
    \vspace{-10pt}
    \begin{tightitemize}
        \item ``CMS 2015 Achievement Award''
        \item Responsible for Operational Aspects of all the Components of the \CreateKeyword{Calorimeter} System
        \item Coordination $\Rightarrow$ Installation $\Rightarrow$ Debugging $\Rightarrow$ DataTaking $\Rightarrow$ Status Report $\Rightarrow$ Collaboration $\Rightarrow$ Training Newcomers
    \end{tightitemize}
} \\
\ExperienceEntry{\textbf{Graduate Research Assistant}}{2014-Current}{\CreateKeyword{CMS} Experiment \CreateKeyword{@CERN}, Geneva, Switzerland \textbullet{} The University of Iowa, Iowa City, IA, USA}{
    \vspace{-10pt}
    \begin{tightitemize}
        \item \CreateKeyword{Big Data} Analyses, e.g. \CreateKeyword{Higgs Boson} Searches. Employed various HPC and HTC facilities (e.g. Fermilab, UI, CERN)
        \item Design, construction and analysis of \CreateKeyword{Monte Carlo} Simulations of Particle Detectors using \CreateKeyword{Geant4}
        \item Data Analysis and Operations Support for \CreateKeyword{Fermilab} T-1041 ``CMS Forward Calorimetry R\&D'' Experiment
    \end{tightitemize}
}

\vspace{10pt}
\Section{Education}
\EducationEntry{\textbf{PhD in Physics}}{2012-2017}{The University of Iowa, Iowa City, IA, USA}{
    \vspace{-10pt}
    \begin{tightitemize}
        \item Thesis Title, ``Search for the Standard Model Higgs Boson in the $\mu^+ \mu^-$ decay channel in pp collisions at $\sqrt{s}=13$ TeV in CMS, Calibration of CMS Hadron Forward Calorimeter and Simulations of Modern Calorimeter Systems''
    \end{tightitemize}
} \vspace{5pt}\\
\EducationEntry{\textbf{BA Physics and Mathematics; Cum Laude}}{2009-2012}{Coe College, Cedar Rapids, IA, USA}{
    \vspace{-10pt}
    \begin{tightitemize}
        \item \CreateKeyword{Minor in Computer Science}
        \item Dean's List Spring 2010 \& Fall 2010
    \end{tightitemize}
} \vspace{5pt}\\
\EducationEntry{\textbf{Department of Cybernetics}}{2008-2009}{Moscow Engineering Physics Institute, Moscow, Russia}{} \vspace{5pt}\\
\EducationEntry{\textbf{Online Education Certificates}}{}{Coursera}{
    \vspace{-10pt}
    \begin{tightitemize}
        \item Functional Programming in Scala
        \item Machine Learning: Regression, Classification
    \end{tightitemize}
}

%\\
%\ExperienceEntry{\textbf{Graduate Teaching Assistant}}{2012-Current}{The University of Iowa, Iowa City, IA, USA}{
%    \vspace{-10pt}
%    \begin{tightitemize}
%        \item Teaching Laboratory and Discussion Sections for General Physics Courses.
%    \end{tightitemize}
%}\\
%\ExperienceEntry{\textbf{Research Assistant}}{2009-2012}{Coe College, Cedar Rapids, IA, USA}{
%    \vspace{-10pt}
%    \begin{tightitemize}
%        \item \CreateKeyword{Raman Spectroscopy} Measurements of Titanium Oxide-based glasses
%        \item \CreateKeyword{Scientific Application Development} for \CreateKeyword{Nuclear Magnetic Resonance Spectroscopy}
%    \end{tightitemize}
%}

%
% Projects
%
%\vspace{10pt}
%\Section{Open Source Activity}
%\ExperienceEntry{\textbf{ROOT4J}}{}{\href{https://github.com/diana-hep/root4j}{https://github.com/diana-hep/root4j}}{
%    \vspace{-10pt}
%    \begin{tightitemize}
%        \item \href{https://root.cern.ch/}{ROOT} I/O for JVM
%    \end{tightitemize}
%}
%\ExperienceEntry{\textbf{Spark-Root}}{}{\href{https://github.com/diana-hep/spark-root}{https://github.com/diana-hep/spark-root}}{
%    \vspace{-10pt}
%    \begin{tightitemize}
%        \item Extension of ROOT4J for use with \href{http://spark.apache.org/}{Apache Spark}
%    \end{tightitemize}
%}

%
% Programming Skills
%
%\vspace{10pt}
\pagebreak
\Section{Programming Skills}
\ProgrammingEntry{\textbf{Languages/etc - Experienced}}{C/C++/STL/Boost/OpenMP/MPI/tbb/CUDA/pthreads/OpenCL/SYCL \textbullet{} Scala \textbullet{} Python \textbullet{} SQL \textbullet{}bash}\\
\ProgrammingEntry{\textbf{Languages/etc - Familiar}}{asm \textbullet{} Java \textbullet{} Go \textbullet{} Rust \textbullet{} php \textbullet{} Exilir/Erlang}\\
\ProgrammingEntry{\textbf{Version Control}}{git/mercury}\\
\ProgrammingEntry{\textbf{Big Data/Machine Learning}}{Apache Spark/MapReduce \textbullet{} HDFS \textbullet{} ROOT \textbullet{} TensorFlow/Keras/Scikit-Learn/Graphlab}\\
\ProgrammingEntry{\textbf{Hardware Design/FPGAs - Familiar}}{Verilog/HLS \textbullet{} Xilinx Vivado/Cadence Xcelium \textbullet{} OpenCL}\\

\vspace{5pt}
\Section{Athletic Activities}
\ExperienceEntry{\textbf{Volunteer Assistant Tennis Coach}}{2013-2014}{The University of Iowa Hawkeyes Men's Tennis Team, NCAA Division 1}{} \vspace{-10pt} \\
\ExperienceEntry{\textbf{Student Athlete}}{2009-2012}{Coe College, Varsity Men's Tennis Team, NCAA Division 3}{
    \vspace{-10pt}
    \begin{tightitemize}
        \item IIAC Team Champion (2012)
        \item NCAA Regionally Ranked in Singles (2011, 2012)
        \item IIAC All-Conference (2009, 2011, 2012)
        \item IIAC Conference Champion (2009, 2010, 2011, 2012)
        \item Team Captain (2011, 2012)
    \end{tightitemize}
}

\vspace{5pt}
\Section{Publications \& Presentations}
\PresentationEntry{\textbullet{} ``CMS Hcal Reconstruction with GPUs''}{\small{``CMS HCAL DPG Meeting'', CERN, Nov. 2019, \href{https://indico.cern.ch/event/858893/contributions/3616860/attachments/1941884/3220135/hcal_dpg_08112019.pdf}{\textbf{presentation link}}}}\\
\PresentationEntry{\textbullet{} ``CMS Ecal Reconstruction with GPUs''}{\small{``CMS ECAL DPG Meeting'', CERN, Oct. 2019, \href{https://indico.cern.ch/event/851469/contributions/3609268/attachments/1932042/3200250/ecal_dpg_23102019.pdf}{\textbf{presentation link}}}}\\
\PresentationEntry{\textbullet{} ``HEP Data Processing with Apache Spark''}{\small{``Spark Summit'', London, Oct. 2018, \href{https://www.youtube.com/watch?v=JTMdkSmw0Kc}{\textbf{youtube talk/presentation}}}}\\
\PresentationEntry{\textbullet{} ``Integrating ROOT I/O with Apache Spark''}{\small{``ROOT Users' Workshop'', Sarajevo, Sep 2018, \href{https://indico.cern.ch/event/697389/contributions/3102790/attachments/1712907/2762150/rootio_apache_spark.pdf}{\textbf{presentation link}}}}\\
\PresentationEntry{\textbullet{} ``spark-root: ROOT I/O for JVM and Applications for Apache Spark''}{\small{``ROOT I/O Workshop'', CERN, Feb 2017}}\\
\PresentationEntry{\textbullet{} ``10B NMR Powder Pattern Optimized for Distribution of the Quadrupole Parameters''}{\small{``Borate 2011: 7th International Conference on Borate Glasses, Crystals and Melts'' Halifax, NS Canada}}\vspace{5pt}
\small{\begin{tightitemize}
    \item A.M. Sirunyan, ..., V. Khristenko et al., ``Search for the Higgs Boson Decaying to Two Muons in Proton-Proton Collisions at $\sqrt{s}=13$ TeV'', Physical Review Letters, 14 January 2019, \url{https://doi.org/10.1103/PhysRevLett.122.021801}
    \item V. Khristenko et al., ``SpectraFit: A New Program to Simulate and Fit Distributed 10B Powder Patterns: Application to Symmetric Trigonal Borons.'', Phys. Chem. Glasses: Eur. J. Glass Sci Technol. B, June 2012, 53 (3), 121-127.
    \item U. Akgun, ..., V. Khristenko et al., ``Characterization of 1800 Hamamatsu R7600-M4PMTs for CMS HF Calorimeter upgrade'', Journal of Instrumentation, 2014 JINST 9 T06005
    \item M. Dettmann, ..., V. Khristenko et al., accepted for publication, ``Radiation Hard Plastic Scintillators for a New Generation of Particle Detectors'', JINST\_023P\_0716
    \item U. Akgun, ..., V. Khristenko et al., ``Quartz Plate Calorimeter Prototype with Wavelength Shifting Fibers'', Journal of Instrumentation, JINST 002P 0412, 2012
    \item A. Albayrak-Yetkin, ..., V. Khristenko ``Secondary Emission Calorimetry: Fast and Radiation-Hard'', Snowmass White Paper, arXiv: 1307.8051.
\end{tightitemize}}



\end{document}
